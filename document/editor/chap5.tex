\chapter{Системийн зохиомж}
\label{chap:intro}
\section{Дэлгэцийн зохиомж}

\munepsfig[width=\textwidth,height=12cm ]{Delgets.jpg}{Хэрэглэгчийн нүүр хуудас }

\pagebreak

\section{Класс диаграм (class diagram)}
\munepsfig[width=\textwidth,height=15cm ]{entityClass.jpg}{ Класс диаграм }
\munepsfig[width=\textwidth,height=15cm ]{boundaryClass.jpg}{ Класс диаграм }
\munepsfig[width=\textwidth,height=15cm ]{controlClass.jpg}{ Класс диаграм }
\munepsfig[width=\textwidth,height=15cm ]{finalClass.jpg}{ Класс диаграм }
\pagebreak

\section{Үйл ажиллагааны диаграм (Activity Diagram) }

\munepsfig[width=\textwidth,height=12cm ]{Activity.jpg}{ Хэрэглэгч зураг засах үйл ажиллагааны диаграм }


\pagebreak

\section{Дарааллын диаграм (Sequence diagram)}
Энэ диаграм нь системийн объектууд хоорондоо хэрхэн харилцдагийг дараалласан зурвас байдлаар тодорхойлох ба энэ зурвасуудад харгалзах объектуудын амьдрах хугацааг үзүүлдэг.

\munepsfig[width=\textwidth,height=12cm ]{Sequence Diagram1.jpg}{зураг засах дарааллын диаграм }



\pagebreak

\section{Төлөв шилжилтийн диаграм (State diagram)}

\munepsfig[width=\textwidth,height=12cm ]{State.jpg}{ Засахын төлвийн диаграм }

\pagebreak



\end{itemize}










