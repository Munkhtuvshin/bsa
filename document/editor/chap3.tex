\chapter{Төслийн хэсэг}
\label{chap:intro}
\section{Системийн хэрэглэгчид}
\begin{itemize}
	\item  хэрэглэгч
	\item  Админ
\end{itemize}
\section{Функционал шаардлага}
\begin{itemize}
	\item  хэрэглэгч
	\item  Админ
\end{itemize}
\begin{itemize}
	\item [1]Зураг Засах
	\item [1.1] Зураг тодоруулна
	\item [1.2] Зурагийн хэмжээ ихэсгэнэ
	\item [1.3] Зурагийн хэмжээ багасгана
	\item [1.4] Эргүүлнэ
	\item [1.5] Тайрах
	\item [1.6]текст бичих
	\item [1.7]Байршуулах
	\item [2] Зураг харна
	\item [3]Илгээх
	\item [4]Устгана
	\item [5]Хариу үйлдэл үзүүлэх(Сэтгэл хөдлөл)
	\begin{itemize}
		\item [1]Бүртгэх
		\item [2]Нэмэх
		\item [3]Хасах
	\end{itemize}
	\section{Функционал бус шаардлага}
	\begin{itemize}
		\item Засах мэдээллийг 5 секундэд хийдэг байх
		\item Системийн загвар хүнд ойлгомжтой байх
		\item Системийг тестэлж, хамгаалсан байх
\end{itemize}
\newpage
\section{ Юз кейс диаграм , түүний тодорхойлолт }

\subsection{Юз кейз  диаграм} 
Энэ диаграм нь ямар функцуудыг гүйцэтгэх, хэтийн төлөв, хэрэгцээ шаардлагуудаас системийг дүрслэн харуулна. 

\munepsfig[width=\textwidth, height=12cm]{Use Case.jpg}{ Юзкейс диаграм }
\newpage

\subsection{ Юз кейз тодорхойлолт }
\begin{center}
	\begin{table}[!htbp]
		\caption{Зураг засах юзкейс диаграмын тодорхойлот}
		\begin{tabular}{|p{4cm}|p{11cm}|}
			\hline
			Товч тайлбар: &Зургийг будах, зураг багагсгах гэх мэт өөр хэрэгтэй байдлаар зургийг өөрчлөх \\
			\hline
			Триггер: & Бүтээгдэхүүн захиалгын модулиудыг ашиглахын тулд системд нэвтрэх шаардлагатай болсон. \\
			\hline
			Үндсэн оролцогч: &  хэрэглэгч \\
			\hline
			Нэмэлт оролцогч: & Байхгүй \\
			\hline
			Өмнөх нөхцөл: &  Зураг засахын өмнө зургийг оруулсан байна\\
			\hline
			Ажлын урсгал: & \begin{enumerate}
				\item Зургийн хэсэгрүү орно.
				\item	Хэрэгсэлүүд ашиглан зургийг өөрт хэрэгтэй байдлаар засварлана
				
			\end{enumerate}
				\\		 
			 \hline
			Дараах нөхцөл: & 	Зураг засагдсан байна.. 	\\	
		   
		  \hline	Альтернатив урсгал: & 	Байхгүй	 
			
			\\	\hline
		\end{tabular}
	\end{table}
\end{center}


\begin{center}
	\begin{table}[!htbp]
		\caption{Зураг илгээх Юзкейз тодорхойлолт}
		\begin{tabular}{|p{4cm}|p{11cm}|}
			\hline
			Нэр: & Зураг илгээх \\
			\hline
			ID: &2\\
			\hline
			Товч тайлбар: &Сошиал дахь найзтайгаа зураг хуваалцахыг хүссэн үед зураг илгээх \\
			\hline
			Триггер: & Системд бүртгэгдээгүй шинэ барааг бүртгэлтэй болгох \\
			\hline
			Үндсэн тоглогч: & хэрэглэгч \\
			\hline
			Нэмэлт тоглогч: & Бусад хэрэглэгч \\
			\hline
			
			Өмнөх нөхцөл: & \begin{enumerate}
				\item Зураг веб-д эсхүл өөрийн компьютерт байна.
			\end{enumerate}	\\
			\hline
			Үндсэн урсгал: & 
			
			\item Зургийн хэсэгрүү орно.
			\item Зургийг илгээх товч дарна.
			\item Зураг илгээх хүнээ сонгоно
			\item Зургаа илгээнэ.
				
			\\		 
			\hline
			Дараах нөхцөл: & Зураг илгээгдсэн байна 	\\	
			
			\hline	Альтернатив урсгал: & Байхгүй 	\\
			\hline
		\end{tabular}
	\end{table}
\end{center}

\begin{center}
	\begin{table}[!htbp]
		\caption{Хариу үйлдэл үзүүлэх (Сэтгэл хөдлөл) Юзкейз тодорхойлолт}
		\begin{tabular}{|p{4cm}|p{11cm}|}
			\hline
			Нэр: & Хариу үйлдэл үзүүлэх (Сэтгэл хөдлөл) \\
			\hline
			ID: &3 \\
			\hline
			Товч тайлбар: &Өөрт таалагдсан болон таалагдаагүй зураг нь дээрээ хариу үйлдэл үзүүлэх \\
			\hline
			Триггер: & Байхгүй \\
			\hline
			Үндсэн тоглогч: & Хэрэглэгч \\
			\hline
			Нэмэлт тоглогч: & Байхгүй \\
			\hline
			
			Өмнөх нөхцөл: & \begin{enumerate}
				\item Зураг системд байршсан байх  
			\end{enumerate}	\\
			\hline
			Үндсэн урсгал: & 
			\item Зургийг үзнэ
			\item Хариу үйлдэл үзүүлнэ..
			
			\\		 
			\hline
			Дараах нөхцөл: & Зураг хариу үйлдэлтэй болно. 	\\	
			\hline	Альтернатив урсгал: & Байхгүй 	\\
			\hline
		\end{tabular}
	\end{table}
\end{center}

\newpage
\section{Ашигласан технологийн шаардлага}

\subsection{Програм ажиллах үйлдлийн системийн сонголт:}

\begin{itemize}
	\item Windows OS: 2007,2008,2010
\end{itemize}

Хамгийн өргөн хэрэглэгддэг нь Microsoft corporation-ийн windows үйлдлийн систем юм.  Windows үйлдлийн системийн давуу талууд :

\begin{itemize}
	\item Хэрэглэхэд хялбар
	\item Асалт илүү хурдан
	\item Бусад техник төхөөрөмжүүдтэй хамгийн сайн зохицдог
	\item Цэнэг (батерей) барилт удаан
	\item Нягтралттай бүхий л дэлгэцэнд тохирох
	\item Windows-ийн бүхий л дагалдах хэрэгсэлтэй холбогдох боломжтой
\end{itemize}

\subsection{Програм хөгжүүлэх хэлний сонголт:}

PHP  програмчлалын хэлийг ашиглан системээ хөгжүүлнэ. PHP програмчлалын хэл нь: 

\begin{itemize}
	\item Сервер талын скрипт хэл
	\item Бүхий л төрлийн үйлдлийн систем/windows, macintosh…/ сервер / Apach, ISS…/ өгөгдлийн сан /MySQL, Oracle, MsSQL, Access…/ дээр ажилладаг.
	\item Сервер дээр файлуудыг нээх, хаах, унших, бичих гэх мэт файлтай ажиллах бүх үйлдлүүдийг хийнэ. 
	\item Хүссэн хуудсандаа хэрэглэгчийн хязгаарлалт хийх боломжтой
	\item Өгөгдлийг хүссэнээрээ нууцалж, кодолж чаддаг
	\item Өгөгдлийн санг үүсгэх, устгах, өгөгдөл нэмэх, устгах, засварлах, хайх гэх мэт үйлдлүүдийг хийдэг
	\item Session, cookie хувьсагчтай ажилладаг
	\item Үнэгүй нээлттэй эх
	\item Динамик веб сайт хийхэд тохиромжтой юм. 
\end{itemize}

\subsection{Өгөгдлийн сангийн удирдах системийн сонголт:}

MySQL өгөгдлийн сан удирдах системийг сонгосон. MySQLнь: 

\begin{itemize}
	\item MySQL нь хамгийн их өргөн тархсан нээлттэй эх код бөгөөд SQL өгөгдлийн сангийн удирдах систем юм. Хүссэн хүн интернэтээс үнэгүй татаж, ашиглах боломжтой. 
	\item Хурдан, найдвартай, ашиглахад хялбар
	\item Их хэмжээний өгөгдөлтэй ажиллах боломжтой/ойролцоогоор 60000 хүснэгт, 50 сая бичлэг/ 
	\item Платформоос үл хамаарч ажиллаж чаддаг бөгөөд бүх платформ дээр ижил үр дүнтэйгээр ажилладаг.
	\item Нээлттэй эх код систем учраас дэлхийн олон мянган мэргэжилтэн хөгжүүлэгч нар алдааг засварлаж, хөгжүүлж байдаг.
	\item C, C++, Eiffel, Java, Perl, PHP, Python, Ruby зэрэг хэлүүдэд зориулан API гаргасан байдаг. Програмчлалын хэлнүүдтэй холбохын тулд ашигладаг Connector/ODBC, Connector/ J, Connector/NET зэрэг холбогчуудыг хувилбар болгон өөр дээрээ гаргадаг. 
	\item Интернэт сүлжээгээр өгөгдлийн санд хандахад холболт, хурд, нууцлалыг тусгаж өгсөн
	\item Олон хэлний горимыг дэмждэг. Алдааны мэдээлэл зэргийг хүссэн хэлээрээ гаргах боломжтой. Өгөгдлөө unicode2-оор хадгалахаас гадна эрэмбэлж болдог. 
\end{itemize}

\subsection{Баримт бичгийн боловсруулалтын системийн сонголт}

Бичиг баримт боловсруулалттай ажлын хувьд:

\begin{itemize}
	\item MS Office Word 2016 
	\item Visual Paradigm for UML 14.1
	\item Enterprise Architecture 
\end{itemize}

Танилцуулах програмын хувьд:

\begin{itemize}
	\item MS Office PowerPoint 2016. 
\end{itemize}



