\documentclass[12pt]{article}
\usepackage[left=12mm,top=0.5in,bottom=5in]{geometry}
\usepackage[utf8]{inputenc} 
\usepackage[T2A]{fontenc} 
\usepackage{enumitem} 
\usepackage{graphicx} %зураг
\usepackage[mongolian]{babel}

\usepackage{anysize}
\marginsize{3cm}{1.5cm}{2cm}{2cm}

\begin{document}
	
\section     {III бүлэг.} Ижил төстэй програмийн судалгаа
	
\section	 {Үүрэг зориулалт}
 
             Анх 2001 оны 2 сарын 4 нд Марк Цукерберг Facebook-ийг Харвардын Их Сургуульд сурдаг хажуу өрөөнийхөө компьютерийн чиглэлээр суралцдаг хоёр найзтайгаа санаачлан хийж байв. Найзуудтайгаа байнга холбоотой байж болохуйц гүүр маягийн вэб сайтыг тэд анх бүтээхээр зорьж байв. Тийм ч учраас 2002 он хүртэл энэхүү сайт нь зөвхөн Харвард Коллежийн дотоод хүрээнд л ажиллаж байв. Ингээд 2002 оны 9 сараас хойш дэлхий дахинаа энэхүү том сүлжээ сайт нь нээгдэж байсан түүхтэй. Энэхүү том сүлжээ сайтийн нэгэн жижиг модул болох санал асуулга буюу (Poll)нь үүрэг зориулалтын хувьд хэрэглэгчийн хувийн чат болон групп чатанд бас ямар нэгэн үйл ажиллагаа явуулах зорилгоор хэрэглэгч үүсгэж ажиллуулах боломжтой бөгөөд үр дүнг тэр дор нь харуулах зориулалттай модул юм.Ямар нэгэн том жижиг асуудлийг олноор түргэн шуурхай шийдэх үр дүнг бий болгож чадна.  
	
	
	
\section     {Зорилго зорилт}

             Энэ модулын зорилго нь цахим сургуулийн орчинд ямар нэгэн хэлэлцэх чувхал асуудлийг шийдвэрлэх олон нийтийн санал хүсэлтийг түргэн шуурхай гаргах дэмжиж буй зүйл дэмжихгүй байгааг өнгө дүрсээр харуулах 
             зорилготой.
             
             
             
             
\section     {IV Бүлэг.} 
             
\section     {Ерөнхий дүгнэлт}  
             
             
             Өнөөгийн мэдээллийн зуун гэж нэрлэгдсэн энэ үед хэн мэдээлэл сайн олж чадаж байна тэр чинээгээрээ амжилт олж чадах болсон. Иймд аливаа байгууллагын үйл ажиллагааг зохион байгуулах, мэдээллийн урсгалыг тодорхой болгож, мэдээллийг хурдан дамжуулах шаардлага зайлшгүй тулгар ч байна. Иймд интернет болон компьютерийн тусламжтай Сургуулийн цахим хуудсийг ажиллуулах нь багш оюутнууд ажил хичээлээ явуулхад тулгарсан ямар нэг асуудлийг шийдхэд нэн чухал үүрэг гүйцэтгэснээр хэрэглэгчийн цаг завийг хэмнэж түргэн шуурхай ажиллаж ямар нэгэн асуудлийг хурдан шуурхай шийдэх боломжийг олгож өгнө гэж ойлгож болох юм.
             
               
             
\section     {Ашигласан материалууд}   
             
             
             1.www.Facebook.com   
             
             2.www.wikipedia  
             
\end{document}             

