\documentclass[12pt]{article}
\usepackage[left=12mm,top=0.5in,bottom=5in]{geometry}
\usepackage[utf8]{inputenc} 
\usepackage[T2A]{fontenc} 
\usepackage{enumitem} 
\usepackage{graphicx} %зураг
\usepackage[mongolian]{babel}

\usepackage{anysize}
\marginsize{3cm}{1.5cm}{2cm}{2cm}

\begin{document}

\section     {IV Бүлэг.} 
    
\section     {Ерөнхий дүгнэлт}  


    Өнөөгийн мэдээллийн зуун гэж нэрлэгдсэн энэ үед хэн мэдээлэл сайн олж чадаж байна тэр чинээгээрээ амжилт олж чадах болсон. Иймд аливаа байгууллагын үйл ажиллагааг зохион байгуулах, мэдээллийн урсгалыг тодорхой болгож, мэдээллийг хурдан дамжуулах шаардлага зайлшгүй тулгар ч байна. Иймд интернет болон компьютерийн тусламжтай Сургуулийн цахим хуудсийг ажиллуулах нь багш оюутнууд ажил хичээлээ явуулхад тулгарсан ямар нэг асуудлийг шийдхэд нэн чухал үүрэг гүйцэтгэснээр хэрэглэгчийн цаг завийг хэмнэж түргэн шуурхай ажиллаж ямар нэгэн асуудлийг хурдан шуурхай шийдэх боломжийг олгож өгнө гэж ойлгож болох юм.
      



               
\section     {Ашигласан материалууд}   

               
             1.www.Facebook.com   
                         
             2.www.wikipedia               
                              
               
\end{document}    